% !TeX root = Project.tex
\section{Introduction} \label{intro}
\subsection{The Definition} \label{sec:def}

The aim of this paper is to eventually define the Lebesgue Integral - so lets try to do that...
\begin{definition}[Lebesgue Integration]
	Given
	\begin{itemize}
		\item a Measure Space; $(X, \A, \mu)$,
		\item an $\A/\eB$ Measurable Function $u\colon X \rightarrow \eR$,
		\item the two positive functions $u^+$ and $u^-$ s.t. $u = u^+ - u^-$,
		\item the two sets $S^+$ and $S^-$ of {\em all} positive Simple Functions ($S^{\pm} \subset \E_\mu^+$) on $(X, \A, \mu)$ s.t. $s \in S^{\pm} \Rightarrow s \leq u^{\pm}$,
	\end{itemize}

	we define the $\mu$-integral of u - which we write as $\int u\,d(\mu)$ - in terms of the integrals of the positive functions $u^+$ and 			$u^-$. 

	The integral of a general {\em positive} function (so including both $u^-$ and $u^+$ but not necessarily $u$ itself) is defined as

	\begin{equation*}\label{posint}\tag{\ref{L}$'$}
	\int u^+\,d(\mu) \coloneqq \sup\bigr\{ I_\mu(s) : \ \  s \in \E^+ \hbox{ \ and \ } s \le u \bigl\},
	\end{equation*}

	and going back to our more general u, 

	\begin{equation}\label{L}
	\int u\,d(\mu) \coloneqq \int u^+\,d(\mu) - \int u^-\,d(\mu).
	\end{equation}

	If the domain of our function is $\R^n$, with the 'standard' $n$-dimensional Borel $\sigma$-algebra and Lebesgue Measure on it - i.e. $(X, \A, \mu) 	= (\R^n, \B, \lambda^n)$ - then this becomes the $\lambda^n$-integral (read as Lebesgue integral instead of lambda integral - or we just 			call it 'the integral') and we write

	\begin{equation*} \tag{\ref{L}$''$}
	\int u\,d(\lambda^n) =  \int u(x)\,d(\Ln) \coloneqq \int u(x)\,d(x) 
	\end{equation*}
	\vskip 5pt
\end{definition}
\subsection{Why Not Just Stop Here?}
That's quite a long definition, one which raises more questions than it answers. The biggest issue here is that even assuming you've taken a course in Analysis, there are many undefined terms;
\begin{itemize}
	\item What are {\em Measures} and {\em $\sigma$-algebras?} And what is a {\em Measure Space}?
	\item What is the {\em Borel $\sigma$-algebra}
	\item What is a {\em Measurable Function\/}? What is a {\em Simple Function}?
	\item What is $I_\mu$ and why can it seemingly only be used on positive {\em Simple Functions}?
\end{itemize}

That last bullet point touches on the second key issue with this definition; why. Why any of this. What about this definition of integration makes it any better than the {\em Riemann Integral}? Do they even describe similar things? This is not {\em at all} immediately obvious, since the Riemann integral is defined on functions $f\colon Y \rightarrow \R$ where Y is some compact\footnote{any closed and bounded interval - so an interval in the form $[a, b]$} subset of $\R$, and this integral is defined on...well, any {\em Measurable Function} on any set at all, apparently.

Even if we can decipher this definition, and then somehow convince ourselves they are compatible, it still seems like a convoluted definition - like, what about Simple Functions is so special that the integrals of all other functions are defined in terms of them (or to be more precise, why is the integral of a function defined in terms of the integrals of positive functions, and then the integrals of positive functions defined in terms of positive simple ones)?

Interestingly (to me, at least) the Riemann integral is defined in a similarish way - we find the upper Riemann integral of a function by finding all the {\em Step Functions} which bound the function from above, define/find the integrals of these using their 'steppyness', then take the infinium - we do the same thing from below, and then if these limits coincided then we call that the Riemann integral. In order to understand the motivations behind Lebesgue Integration better, let's take a slightly closer look at the similarities and differences between these two.

\begin{abstract}
\vskip 3mm
In this paper we work our way up to defining the {\bf\it Lebesgue Intergral\/} by introducing relevent results from Set Theory and Measure Theory. We make comparisions between the Lebesgue Intergral and the Reimann Intergral, noting the relative stregths and weaknesses of each. 
\vskip 2mm
Over the course of the paper, we will touch on other uses of Measures Theory, including it's use in Axiomatic Set Theory in the definition of a random variable. Finally, we will look very briefly at $\Lb ^p$ spaces; we'll simply state what they are and make vauge but interesting statements about what it means for the $\Lb ^p$s to be normed vector spaces, and that $\Lb ^2$ in particular is an inner product space. 
\end{abstract}
\subsection{The Motivation}

Our goal when we integrate a function is to find the 'space' or area between the graph and  x-axis (or the Abscissa - but I'll just say x-axis). It is difficult to say how much space an arbitrary shape takes up, but we can work out the area of rectangles very easily - by calculating the (base $\times$ height). 

*Picture of approximating a graph with rectangles*

In the case where the domain of the function (the 'x-axis'), is $\R^2$, we are actually calculating a volume by finding the 'space' between the graph - still, the 'area of a rectangle' (in this case, the volume of a cuboid) is calculated in much the same way; the size of the base (which is now the area of the base) $\times$ the height. 

*Picture of the appromimatimg a 3D graph with cuboids*

With this is mind, lets clear up some of the terms we are going to use:

{\bf \em Lebesgue integration\/} is the method of integration defined above. Just like Rieman Interation, it is used to find the space between the graph of a function and the domain - however, the domain can be any set at all. Thinking back to rectangles and cuboids (size of base $\times$ height), if we wanted make sense of what it means for there to be space between the graph and the x-axis, we should be able to make sense of the 'size' of parts of the domain.

{\bf \em The Lebesgue Measure\/} seems like it'd then be the technique for assigning sizes to, or 'measuring' aribtray sets; but it is not! It's specifically for when we give sizes to the subsets of $\R^n$, and ones which coicied with our usual idea of the size of a set in $\R^n$ - i.e. it is used when we are considering functions $\R^n \supset Y \rightarrow \R$. 

{\bf \em The Lebesgue Integral\/} is what we get when we do Lebesgue Intergration on a set with the Lebesgue Measure on it. i.e. it's the intergral of functions $\R^n \supset Y \rightarrow \R$. This is the (basically) the same domain as the Riemann integral and so we'd want to check that these two things match up - and then see if the Lebesgue integral is better in some way.

To see the motivation behind the Lebesgue integral over Riemann, let's look at the canonical example of a function for which the Reimann integral fails; $f:\R \rightarrow \R,\ \ f(x) = \mathbbm{1}_\Q(x)$.

*Picture of this function*

First, let's define the Riemann integral. The definition of the Riemann integral involes many steps, and so it's quite long.
First, lets give the definition of a step function;
\begin{definition}{{(\em Riemann Intergration\/})}
	Given;
	\begin{itemize}
		\item A function, $f: \ \R \rightarrow \R$
		\item A
	\end{itemize}
\end{definition}
	




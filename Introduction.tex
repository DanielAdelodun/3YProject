\section{Introduction} \label{intro}
\subsection{The Definition} \label{def}

The aim of this paper is to eventually define the Lebesgue Intergral - so lets just do exactly that:
\begin{definition}
	Given
	\begin{itemize}
		\item a Measure Space; $(X, \A, \mu)$,
		\item an $\A/\eB$ Measuarable Function $u : X \rightarrow \eR$,
		\item positive functions $u^+$ and $u^-$ s.t. $u = u^+ - u^-$,
		\item the sets $S^{\pm}$ of {\em all} Simple Functions ($S \subset \E_\mu^+$) on $(X, \A, \mu)$ s.t. $s \in S^{\pm} \Rightarrow s \leq u^{\pm}$,
	\end{itemize}

	we define the $\mu$-integral of u - which we write as $\int u\,d(\mu)$ - in terms of the integrals of the positive functions $u^+$ and 			$u^-$. The intergral of a general positive function - so including our $u^{\pm}$ - is defined as

	\begin{equation*}\label{posint}\tag{\ref{L}$'$}
	\int u^+\,d(\mu) \coloneqq \sup\bigr\{ I_\mu(s) : \ \  s \in \E^+ \hbox{ \ and \ } s \le u \bigl\},
	\end{equation*}

	and going back to our more general u, 

	\begin{equation}\label{L}
	\int u\,d(\mu) \coloneqq \int u^+\,d(\mu) - \int u^-\,d(\mu).
	\end{equation}

	If the domain of our function is $\R^n$, with the 'standard' $n$-dimensional Borel $\sigma$-algebra and Lebesgue Measure on it - i.e. $(X, \A, \mu) 	= (\R, \B, \lambda^n)$ - then this becomes the $\lambda^n$-integral (read as Lebesgue integral instead of lambda integral - or we just 			call it 'the integral') and we write

	\begin{equation*} \tag{\ref{L}$''$}
	\int u\,d(\lambda^n) =  \int u(x)\,d(\Ln) \coloneqq \int u(x)\,d(x) 
	\end{equation*}
\end{definition}

That's quite a long definition, one which raises more questions than it answers. The first (and biggest) issue here is that, even assuming someone has taken a course in Analysis, there are many undefined terms;
\begin{itemize}
	\item What are Measures and $\sigma$-algebras? And what is a Measure Space?
	\item What is the Borel $\sigma$-algerbra
	\item What is a Measuarable Function? What is a Simple Function?
	\item What is $I_\mu$ and why can it seemingly only be used on positive Simple Functions?
\end{itemize}

That last bullet point touches on the second key issue with this definition; why. Why any of this. What about this definition of what we call the integral makes it any better than the Reimann Integral? Do they even describe similar things? This is not at all immedately obvious, since the Reimann intergral is defined on functions $f: Y \rightarrow \R$ where Y is some compact\footnote{Define a compact set} subest of R and this integral is defined on...well, any set at all.

Even if we can somehow convince ourselves they are compatible, it still seems like a convoluted defintion - what about Simple Functions is so special that the intergrals of all other functions are defined in terms of them (or to be more precise, why are general functions are defined in terms of positive functions, and then positive functions defined in terms of positive simple ones)?

Interestingly (to me, at least) the Reinmann intergral is defined in a similarish way - we find the upper Reimann intergral of a function by finding all the Step Functions which bound the function from above, define/find the intergrals of these using their 'steppyness', then take the infinium - we do the same thing from below, and then if these limits coiencied then we call that the Reinmann integral. In order to understand the motivations behind Lebegsue Intgration better, let's take a slightly closer look at the similarites and differences between these two.
\newpage
\subsection{The Motivation}

Our goal when we integrate a function is to find the amount of `space' between the graph and x-axis (or the Abscissa if you're fancy - but I'll just say x-axis). It's pretty difficult to say how much space an arbitrary curve takes up, but we can work out the area of rectangles very easily - by calculating the {\em base $\times$ height}. Riemann integration takes advantage of this, and defines the integral of a function $f:\R \rightarrow \R$ by first bounding it from above with rectangular functions and finding the smallest of these areas (which is would be an upper bound for the area of the function), then we bound the function from below and find an lower bound, then we hope that these two bounds match up. 

\begin{figure}[H]
\centering
\begin{minipage}{.5\textwidth}
  \centering
  \includegraphics{Code/Area1.png}
  \captionof{figure}{Bounding from Below}
  \label{fig:test1}
\end{minipage}%
\begin{minipage}{.5\textwidth}
  \centering
  \includegraphics{Code/Area2.png}
  \captionof{figure}{Bounding from Above}
  \label{fig:test1}
\end{minipage}
\end{figure}

If the domain of our function (the `x-axis'), is $\R^2$ instead of just $\R$, we'd actually calculating a volume by finding the `space' between the graph and the axis. Basic Riemann integration doesn't apply to these functions - still, you can think of the `area of a rectangle' (which in $R^2 \rightarrow \R$ would actually be the volume of a cuboid) as being calculated in much the same way; the size of the base (which is now an area, instead of a length) $\times$ the height. With this is mind, lets clear up some of the terms we are going to use:

\begin{description}
\item[\em Lebesgue integration\/] is the method of integration defined above. Just like Riemann Integration, it is used to find the space between the graph of a function and the domain. However, unlike Riemann, the domain can be any set at all. Thinking back to rectangles and cuboids and $base \times height$, if we wanted make sense of what it means for there to be space between the graph and the x-axis, we should be able to make sense of the `size' of parts of the domain, so that we can figure out how big our base is and multiply that by the height$\ldots$
%
\item[\em The Lebesgue Measure\/] seems then like it'd be a way to assign sizes to, or `measure' sets in the domain; but it is not! It's a term used specifically when we give sizes to the subsets of $\R^n$, and those sizes (or `measures') coincided with our usual idea of the size of a set in $\R^n$ - i.e. it is used when we are considering functions $\R^n \supset Y \rightarrow \R$, and in $\R$ the measure (length) of the interval $[1, 0]$ is 1, in $\R^2$ the measure (area) of the rectangle $[1, 0] \times [1, 0]$ is 1, in $\R^3$ the measure (volume) of the cube $[1, 0] \times [1, 0] \times [1, 0]$ is 1$\ldots$ etc. 
%
\item[\bf \em The Lebesgue Integral\/] is what we get when we do Lebesgue Integration on a set with the Lebesgue Measure on it. i.e. it's the integral of normal functions $\R^n \supset Y \rightarrow \R$. This is the (basically) the same domain as the Riemann integral and so we'd want to check that the Lebesgue Integral and the Riemann integral match up - and then see if the Lebesgue integral is better in some way.

To see the motivation behind the Lebesgue integral over Riemann, let's look at the canonical example of a function for which the Riemann integral fails; $f:\R \rightarrow \R,\ \ f(x) = \mathbbm{1}_\Q(x)$. This is also an example of a {\em Simple Function} since it is just the indicator function of a measurable set, but lets not get ahead of ourselves...
\end{description}
\begin{figure}[H]
	\centering
	\includegraphics[]{Code/Rational.png}
	\caption{An illustration of the above indicator function, showing a few rationals that are multiples of negative powers of 2.}
\end{figure}

First, let's define the Riemann integral. The definition of the Riemann integral involves a few steps, and so it's going to take a little patience. The first step is defining a step function (no pun intended).

A function is called a step function if and only if there exists some finite sequence of points (which we can index by n and call $x_n$) such that the function is constant between any two adjacent points. So given the function with the graph;
\begin{figure}[H]
	\centering
	\includegraphics{Code/Step.png}
	\caption{A Step Function}
\end{figure}
the points \{0, 2, 3, 6, 8, 10\} are the finite sequence which make this a step function. Note two important facts;
\begin{itemize}
	\item We could have picked the points \{0, 1, 2, 3, 4, 5, 6, 7, 8, 9, 10\} to be our finite sequence, as the above function also happens to constant between any two adjacent integers. 
	\item It doesn't really matter what the function is {\em at} the points $x_n$, since we only care whether the function is constant {\em between} the points. In our graph, 2, 3, 6, and 8 are discontinues, but that doesn't matter since they're also in our sequence of points. 
\end{itemize}

The formal definition is as follows;
\begin{definition}{{(\bf\em Step Function\/})}
, $f: \ [a, b] \rightarrow \R$ is a step function if and only if there exists $n \in  \N$ and $a = x_0$
\end{definition}




% !TeX root = Project.tex
\clearpage
\noappendicestocpagenum
\begin{appendices}
\section{List of Definitions}
Here I list some definitions which I did not have time to work into the main body of the text. I skip some definitions that you'd expect most maths students to know. The notation here mostly follows the style documented at the \href{https://en.wikipedia.org/wiki/List_of_mathematical_symbols}{Wikipedia} page for mathematical notation. Sets are normally denoted with capitals, $A, B, C\ldots$, and sets of [sub]sets in calligraphic font $\A, \B, \mathcal{C}\ldots$ Notably, the power set $\mathcal{P}(X)$ of a set X follows this pattern.
% !TeX root = Project.tex

% 1
\begin{definition}[The Extended Real Line, $\eR$]
$$\eR \defeq \R \cup \{-\infty,\> +\infty\}$$
\end{definition}

% 2
\begin{definition}[$\sigma$-Algebra]
Given
\begin{itemize}
\item
	A set $X$,
\item
	$\A \subset \P(X)$.
\end{itemize}
Then $\A$ is a \emph{$\sigma$-Algebra} of $X$ $\logeq$
\begin{enumerate}[(i)]
\centering
\item
	$\emptyset \in \A$ \vspace{2pt},
\item
	$(A_j)_{j\in\N} \subset \A \Rightarrow \bigcup\limits_\N A_j \in \A$,
\item
	$A \in \A \Rightarrow  X\backslash{}A [\ \equiv A^c] \in \A$.
\end{enumerate}
\end{definition}

% 3
\begin{definition}[Measurable Space]
Given
\begin{itemize}
\item
	A set $X$,
\item
	A $\sigma$-Algebra of $X$, $\A$.
\end{itemize}
Then
$$\text{\it The ordered pair }(X, \A) \text{\it \ is a } \emph{Measurable Space}.$$
\end{definition}

% 4
\begin{definition}[The $\sigma$-Algebra Generated by $\G, \sigma(\G)$]
Given 
\begin{itemize}
\item
	A set $X$,
\item
	$\G \subset \P(X)$.
\end{itemize}
Then $\sigma(\G)$ is the \emph{$\sigma$-Algebra Generated by $\G$} $\logeq$
\begin{enumerate}[(i)]
\centering
\item
	$\G \subset \sigma(\G)$,
\item
	$\sigma(\G) \text{\it\ is a } \sigma\text{\it -Algebra}$,
\item
	$\A \text{\it\ is a }\sigma\text{\it-Algebra of } X \text{\it\ and } \G \subset \A  \Rightarrow \sigma(\G) \subset \A.$
\end{enumerate}
\end{definition}

% 5
\begin{definition}[The family of half open rectangles in $\R^n$, $\J$]
$$\P(\R^n) \supset \J \defeq \{(a_1, b_1] \times (a_2, b_2] \times \dots \times (a_n, b_n]; \; \forall a_k, b_k \in \R\}$$
\end{definition}

% 6
\begin{definition}[The Borel $\sigma$-Algebra on $\R^n, \B(\R^n)$]
Fix $n \in \N$. Given the family of half open rectangles $\mathcal{J}^{o,n}$
$$ \text{\it The } \sigma\text{\it -Algebra generated by } \mathcal{J}^{o,n}\text{\it\ is the Borel }\sigma\text{\it-Algebra on }\R^n$$
\end{definition}

% 7
\begin{definition}[The Extended Borel $\sigma$-Algebra on $\eR^n, \eB(\eR^n)$]
Fix $n \in \N$. $S^* \in \eB \logeq$
$$\exists B \in \B(\R) \text{\it\ and }\exists S \in \{\emptyset,\> +\infty,\> -\infty, \{+\infty, -\infty\}\}: \quad S^* = B \cup S$$
\end{definition}

% 8
\begin{definition}[Measures]
Given 
\begin{itemize}
\item
	A Measurable Space $(X, \A)$,
\item
	A map $\mu\colon \A \rightarrow \R$.
\end{itemize}
$\text{Then } \mu \text{ is a} \emph{ Measure } \text{on the Measurable Space } (X, \A) \logeq$
\begin{enumerate}[(i)]
\centering
\item
	$\mu(\emptyset) = 0$
\item
	$A, B \in \A; \> A \cap B  = \emptyset \quad \Rightarrow \quad \mu(A \cup B) = \mu(A) + \mu(B)$
\end{enumerate}
\end{definition}

% 9
\begin{definition}[Measure Space]
Given
\begin{itemize}
\item
	A Measurable Space $X$,
\item
	A Measure $\mu$ on $(X, \A)$.
\end{itemize}
Then
$$\text{\it The ordered triple }(X, \A, \mu) \text{\it \ is a } \emph{Measure Space}.$$
\end{definition}


% 10
\begin{definition}[{[Discrete]} Probability Space]
Given
\begin{itemize}
\item[\tiny$\otimes$]
	$(\omega_n)_{n \in \N} = \Omega$ [\emph{for a Discrete Probability Space}],
\item
 	A Measure Space $(\Omega, \A, p)$.
\end{itemize}
Then $(\Omega, \A, p)$ is a \emph{[Discrete] Probability Space} and $p$ a \emph{Probability Measure} $\logeq$
\begin{itemize}
\centering
\item[]
	$p(\Omega) = 1$.
\item[]
	(ii*) \ \ $\bigcup\limits_\N \mu(\{\omega_n\}) = 1$ \emph{[Discrete]}.
\end{itemize}
\end{definition}

% 11
\begin{definition}[The Lebesgue Measure $\lambda^n$]
Fix $n \in \N.$ Given 
\begin{itemize}
\item
	The Measurable Space $\bigl(\R^n, \B(\R^n)\bigr)$,
\item
	A Measure $\lambda^n$ on $\bigl(\R^n, \B(\R^n)\bigr)$.
\end{itemize}
Then $\lambda^n$ is the \emph{Lebesgue Measure} on $\bigl(\R^n, \B(\R^n)\bigr) \logeq$ 
$$\forall X \in \J\!, \quad \lambda^n(X) = \prod_{k=0}^n(b_k - a_k).$$
\end{definition}

% 12
\begin{definition}[Measurable Mapping]
Given
\begin{itemize}
\item
	Measurable Spaces $(X, \A)$ and $(X', \A')$,
\item
	A map $T\colon X \rightarrow X'$.
\end{itemize}
Then T is an \emph{$(\A/\A')$-Measurable Map} $\logeq$
$$ A' \in \A' \Rightarrow T^{-1}(A') \in \A.$$
\end{definition}

% 13
\begin{definition}[Image Measure]
Given
\begin{itemize}
\item
	Measurable Spaces $(X, A), \> (X', \A')$,
\item
	A Measurable Mapping, $T\colon X \rightarrow X'$,
\item
	An arbitrary Measure $\mu$ on $(X, A)$.
\end{itemize}
Then,
$$\mu'\colon \A' \rightarrow \R, \>\> A \mapsto \mu\bigl(T^{-1}(A')\bigr) \text{\it\ is the Image Measure of $\mu$ under $T$}.$$
\end{definition}

% 14
\begin{definition}[Measurable {[Numerical]} Function]
Fix $n \in \N$. Given
\begin{itemize}
\item
	Measure Spaces $(X, \A), \> (X', \A')$,
\item
	An $(\A/\A')$-Measurable Mapping, $u\colon X \rightarrow X'$.
\end{itemize}
Then, if $(X', \A') = \bigl(\R^n, \B(\R^n)\bigr) \>\, [\,\bigl(\eR^n, \eB(\eR^n)\bigr)\,]$
$$u \text{\it\ is a Measurable [Numerical] Function, } u\colon X \rightarrow \R^n \> [\eR^n]$$
\end{definition}

%%% 15 %%%
\begin{definition}[{[Real-Valued]} Random Variable]
Given 
\begin{itemize}
\item
A Probability Space $(\Omega, \A, p)$ and a Measurable Space $(X[=\R], \A')$,
\item
An ($\A/\A'$)-Measurable Mapping $Y\colon\Omega \rightarrow X[= \R]$.
\end{itemize}
Then $Y$ is a \emph{[Real-Valued] Random Variable}.  The Image Measure of $p$ on $Y$ gives the \emph{Probabilities} of the events $E \in A'.$ 
\end{definition}

% 15
\begin{definition}[Simple Function]
Given
\begin{itemize}
\item
	A Measure Space $(X, \A)$ along with $\measurespace$,
\item
	A Measurable Function $u: X \rightarrow \R.$
\end{itemize}
Then $u$ is a \emph{Simple Function} $\logeq$
$$\exists N \in \N: \> \exists(A_j)_{j=0}^N \text{it\ with } A_j \in \A \text{\it\ and } \> \exists(c_j)_{j=0}^N \text{\it\ with } %
c_j \in \R \> : u(x) = \sum_{j=0}^N c_j \cdot \indi[A_j] \;\; \forall x \in X$$
\end{definition}

% 16
\begin{definition}[Integral of a Simple Function]
Given
\begin{itemize}
\item
	A Measure Space, $(X, \A, \mu)$,
\item
	A Simple Function, $s\colon X \rightarrow \R$,
\item
	$N \in \N$, and sets $(A_j)_{j=0}^N$ and $(c_j)_{j=0}^N : s(x) = \sum_{j=0}^N c_j \cdot \indi[A_j] \;\; \forall x \in X$.
\end{itemize}
Then $I_\mu$ is the \emph{Integral of the Simple Function} $s$, where
$$I_\mu = \sum_{j=0}^N c_j \cdot \mu(A_j)$$
\end{definition}
\end{appendices}
% !TeX root = Project.tex
\section{Introduction}
%
%
\subsection{The Definition}\label{def:main}
The aim of this paper is to eventually define the Lebesgue Integral - so lets try to do that...
\begin{definition}[Lebesgue Integration]
	Given
%
	\begin{itemize}
		\item A \emph{\dref{def:mspace}[Measure Space]}, $(X, \A, \mu)$,
		\item An \dref{def:mfun}[$(\A/\eB)$\emph{-Measurable Function}], $u\colon X \rightarrow \eR$,
		\item $u^+, u^- : u = u^+ - u^- \text{ and }u^{\pm}(x) > 0 \;\; \forall x \in X$,  
		\item $S^+, S^- \subset \dref{def:sfun}[\E_\mu^+] : s \in S^{\pm} \Leftrightarrow s \leq u^{\pm}, \text{ and } s \in \dref{def:sfun}[\E^+_\mu]$.
	\end{itemize}
%
	We define the {\color{Magenta}\emph{$\mu$-Integral}} of $u$ --- $\int u\,d(\mu)$ --- in terms of the {\color{Magenta}\emph{$\mu$-Integrals}}
	of the positive functions $u^+$ and $u^-$. 

	The {\color{Magenta}\emph{$\mu$-Integral}} of a {\em positive} function (so including both $u^-$ and $u^+$ but not necessarily $u$ itself) is defined as
%
	\begin{equation*}\label{posint}\tag{\ref{L}$'$}
	{\color{Magenta}\int u^+\,d(\mu)} \defeq \sup\bigr\{ \dref{def:sint}[I_\mu(s)] : \ \  s \in \E^+ \text{ \ and \ } s \le u^+ \bigl\},
	\end{equation*}
%
	and going back to our more general u, 
%
	\begin{equation}\label{L}
	{\color{Magenta}\int u\,d(\mu)} \defeq \int u^+\,d(\mu) - \int u^-\,d(\mu).
	\end{equation}

	If the domain of our function is $\R^n$, with the 'standard' $n$-dimensional \dref{def:balgebra}[\emph{Borel $\sigma$-Algebra}] and 
	\dref{def:lmeasure}[\emph{Lebesgue Measure}] on it --- i.e. $(X, \A, \mu) = (\R^n, \B, \lambda^n)$ --- then this becomes the {\color{Magenta}\emph{$\lambda^n$-Integral,}}
	 (read as {\it \color{Magenta}Lebesgue Integral} instead of lambda integral --- or we just call it the {\color{Magenta}\emph{Integral}}, confusingly) and we write
%
	\begin{equation*} \tag{\ref{L}$''$}
	\int u\,d(\lambda^n) =  \int u(x)\,d(\Ln) \eqdef {\color{Magenta}\int u(x)\,dx}. 
	\end{equation*}
	\vskip 5pt
\end{definition}

\subsection{Why Not Just Stop Here?}
That's quite a long definition, one which raises more questions than it answers. The biggest issue here is that, even assuming you've taken a course in Analysis, there are many undefined terms;
\begin{itemize}
	\item What are \dref{def:measure}[\emph{Measures}] and \dref{def:balgebra}[\emph{$\sigma$-Algebras}] And what is a \dref{def:measure}[\emph{Measure Space}]?
	\item What is the \dref{def:balgebra}[\emph{Borel $\sigma$-algebra}]
	\item What is a \dref{def:mfun}[\emph{Measurable Function}]? What is a \dref{def:sfun}[\emph{Simple Function}]?
	\item What is \dref{def:sint}[\emph{$I_\mu$}] and why can it seemingly only be used on positive \dref{def:sfun}[\emph{Simple Functions}]?
\end{itemize}

That last bullet point touches on the second key issue with this definition; why. Why any of this. What about this definition of integration makes it any better than the \dref{def:riemann}[\emph{Riemann Integral}]? Do they even describe similar things? This is not {\em at all} immediately obvious, since the Riemann Integral is defined on functions $f\colon Y \rightarrow \R$ where Y is some compact\footnote{any closed and bounded interval - so an interval in the form $[a, b]$} subset of $\R$, and this Integral is defined on...well, \dref{def:mfun}[\emph{Measurable Functions}] on any set at all, apparently.

Even if we can decipher this definition, and then somehow convince ourselves it's compatible with Riemann's definition, it still seems a little convoluted - like, what about \dref{def:sfun}[\emph{Simple Functions}] is so special that the Integrals of all other functions are defined in terms of them. Why is the Integral of a function defined in terms of the Integrals of positive functions, while the Integral of a positive function is defined in terms of positive simple ones?

Interestingly (to me, at least) the \dref{def:riemann}[\emph{Riemann Integral}] is defined in a similarish way - we find the upper Riemann Integral of a function by finding all the \dref{def:stfun}[\emph{Step Functions}] which bound the function from above, define/find the \dref{def:stint}[\emph{Integrals}] of these using their 'steppyness', then take the infinium - we do the same thing from below, and then if these limits coincided then we call that the Riemann Integral. In order to understand the motivations behind Lebesgue Integration better, let's take a slightly closer look at the similarities and differences between these two.

% !TeX root = Project.tex

\section{List of Definitions}
Here I list some definitions which I did not have time to work into the main body of the text. I skip some definitions that you'd expect most maths students to know. The notation here mostly follows the style documented at the \href{https://en.wikipedia.org/wiki/List_of_mathematical_symbols}{Wikipedia} page for mathematical notation. Sets are normally denoted with capitals, $A, B, C\ldots$, and sets of [sub]sets in calligraphic font $\A, \B, \mathcal{C}\ldots$ Notably, the power set $\mathcal{P}(X)$ of a set X follows this pattern.

% 1
\begin{definition}[The Extended Real Line, $\eR$]\label{def:eRealLine}
$${\color{Magenta}\eR} \defeq \R \cup \{-\infty,\> +\infty\}$$
\end{definition}

% 2
\begin{definition}[$\sigma$-Algebra]\label{def:salgebra}
Given
\begin{itemize}
\item
	A set, $X$,
\item
	A family of subsets of $X$, $\A \subset \P(X)$.
\end{itemize}
Then $\A$ is a {\color{Magenta}\emph{$\sigma$-Algebra}} of $X$ $\logeq$
\begin{enumerate}[(i)]
\centering
\item
	$\emptyset \in \A$ \vspace{2pt},
\item
	$(A_j)_{j\in\N} \subset \A \Rightarrow \bigcup\limits_\N A_j \in \A$,
\item
	$A \in \A \Rightarrow  X\backslash{}A [\ \equiv A^c] \in \A$.
\end{enumerate}
\end{definition}

% 3
\begin{definition}[Measurable Space]\label{def:mablespace}
Given
\begin{itemize}
\item
	A set, $X$,
\item
	A \dref{def:salgebra}[\emph{$\sigma$-Algebra}] of $X$, $\A$.
\end{itemize}
Then
\begin{itemize}
\item
	$\text{\it The ordered pair }(X, \A) \text{\it \ is a {\color{Magenta}Measurable Space}}$.
\item
	$\text{ A set }\A \in \A \text{\it\ is {\color{Magenta}Measurable}}.$
\end{itemize}
\end{definition}

% 4
\begin{definition}[The $\sigma$-Algebra Generated by $\G, \sigma(\G)$]
Given 
\begin{itemize}
\item
	A set, $X$,
\item
	$\G \subset \P(X)$.
\end{itemize}
Then {\color{Magenta}$\sigma(\G)$} is the {\color{Magenta}\emph{$\sigma$-Algebra Generated by $\G$}} $\logeq$
\begin{enumerate}[(i)]
\centering
\item
	$\G \subset {\color{Magenta}\sigma(\G)}$,
\item
	${\color{Magenta}\sigma(\G)} \text{\it\ is a } \dref{def:salgebra}[\sigma\text{\it -Algebra}] \text{\it\ of }X$,
\item
	$\A \text{\it\ is a }\dref{def:salgebra}[\sigma\text{\it -Algebra}] \text{\it\ of } X \text{\it\ and } \G \subset \A  \Rightarrow {\color{Magenta}\sigma(\G)} \subset \A.$
\end{enumerate}
\end{definition}

% 5
\begin{definition}[The family of half open rectangles in $\R^n$, $\J$]\label{def:rect}
$$\P(\R^n) \supset {\color{Magenta}\J} \defeq \{(a_1, b_1] \times (a_2, b_2] \times \dots \times (a_n, b_n]; \; \forall a_k, b_k \in \R\}$$
\end{definition}

% 6
\begin{definition}[The Borel $\sigma$-Algebra on $\R^n, \B(\R^n)$]\label{def:balgebra}
Fix $n \in \N$. Given the family of half open rectangles $\mathcal{J}^{o,n}$
$$ \text{\it The } \dref{def:salgebra}[\sigma\text{\it -Algebra}] \text{\it\ generated by } {\color{Magenta}\mathcal{J}^{o,n}}\text{\it\ is the } \text{\color{Magenta}\it Borel $\sigma$-Algebra} \text{\it\ on }\R^n$$
\end{definition}

% 7
\begin{definition}[The Extended Borel $\sigma$-Algebra on $\eR^n, \eB(\eR^n)$]
Fix $n \in \N$. $S^* \in {\color{Magenta}\eB} \logeq$
$$\exists B \in \B(\R) \text{\it\ and }\exists S \in \{\emptyset,\> +\infty,\> -\infty, \{+\infty, -\infty\}\}: \quad S^* = B \cup S$$
\end{definition}

% 8
\begin{definition}[Measures]\label{def:measure}
Given 
\begin{itemize}
\item
	A \dref{def:mablespace}[\emph{Measurable Space}], $(X, \A)$,
\item
	A map, $\mu\colon \A \rightarrow \R$.
\end{itemize}
$\text{Then } {\color{Magenta}\mu} \text{ is a} {\color{Magenta}\emph{ Measure }} \text{on the \dref{def:mablespace}[\emph{Measurable Space}] } (X, \A) \logeq$
\begin{enumerate}[(i)]
\centering
\item
	${\color{Magenta}\mu}(\emptyset) = 0$
\item
	$A, B \in \A; \> A \cap B  = \emptyset \quad \Rightarrow \quad {\color{Magenta}\mu}(A \cup B) = {\color{Magenta}\mu}(A) + {\color{Magenta}\mu}(B)$
\end{enumerate}
\end{definition}

% 9
\begin{definition}[Measure Space]\label{def:mspace}
Given
\begin{itemize}
\item
	A \dref{def:mablespace}[\emph{Measurable Space}], $X$,
\item
	A \dref{def:measure}[\emph{Measure}], $\mu$, on $(X, \A)$.
\end{itemize}
Then
$$\text{\it The ordered triple }(X, \A, \mu) \text{\it \ is a } {\color{Magenta}\emph{Measure Space}}.$$
\end{definition}


% 10
\begin{definition}[{[Discrete]} Probability Space]\label{def:pspace}
Given
\begin{itemize}
\item[\tiny$\otimes$]
	$(\omega_n)_{n \in \N} = \Omega$ [\emph{for a Discrete Probability Space}],
\item
 	A \dref{def:mspace}[\emph{Measure Space}], $(\Omega, \A, p)$.
\end{itemize}
Then $(\Omega, \A, p)$ is a {\color{Magenta}\emph{[Discrete] Probability Space}} and {\color{Magenta}$p$} a {\color{Magenta}\emph{Probability Measure}} $\logeq$
\begin{itemize}
\centering
\item[]
	$p(\Omega) = 1$.
\item[]
	(ii*) \ \ $\bigcup\limits_\N \mu(\{\omega_n\}) = 1$ \emph{[Discrete]}.
\end{itemize}
\end{definition}

% 11
\begin{definition}[The Lebesgue Measure $\lambda^n$]\label{def:lmeasure}
Fix $n \in \N.$ Given 
\begin{itemize}
\item
	The \dref{def:mablespace}[\emph{Measurable Space}], $\bigl(\R^n, \B(\R^n)\bigr)$,
\item
	A \dref{def:measure}[\emph{Measure}], $\lambda^n$, on $\bigl(\R^n, \B(\R^n)\bigr)$.
\end{itemize}
Then {\color{Magenta}$\lambda^n$} is the {\color{Magenta}\emph{Lebesgue Measure}} on $\bigl(\R^n, \B(\R^n)\bigr) \logeq$ 
$$\forall X \in \J\!, \quad {\color{Magenta}\lambda^n(X)} = \prod_{k=0}^n(b_k - a_k).$$
\end{definition}

% 12
\begin{definition}[Measurable Mapping]\label{def:mmap}
Given
\begin{itemize}
\item
	\dref{def:mablespace}[\emph{Measurable Spaces}], $(X, \A)$ and $(X', \A')$,
\item
	A map, $T\colon X \rightarrow X'$.
\end{itemize}
Then {\color{Magenta}$T$} is an {\color{Magenta}\emph{$(\A/\A')$-Measurable Map}} $\logeq$
$$ A' \in \A' \Rightarrow T^{-1}(A') \in \A.$$
\end{definition}

% 13
\begin{definition}[Image Measure]\label{def:imeasure}
Given
\begin{itemize}
\item
	\dref{def:mablespace}[\emph{Measurable Spaces}], $(X, A), \> (X', \A')$,
\item
	A \dref{def:mmap}[\emph{Measurable Mapping}], $T\colon X \rightarrow X'$,
\item
	An \dref{def:measure}[\emph{Measure}], $\mu$, on $(X, A)$.
\end{itemize}
Then,
$${\color{magenta}\mu'\colon} \A' \rightarrow \R, \>\> A \mapsto \mu\bigl({T}^{-1}(A')\bigr) \text{\it\ is the {\color{Magenta}Image Measure} of $\mu$ under {$T$}}.$$
\end{definition}

% 14
\begin{definition}[Measurable {[Numerical]} Function]\label{def:mfun}
Fix $n \in \N$. Given
\begin{itemize}
\item
	\dref{def:mspace}[\emph{Measure Space}], $(X, \A), \> (X', \A')$,
\item
	An \dref{def:mmap}[\emph{$(\A/\A')$-Measurable Mapping}], ${\color{Magenta}u}\colon X \rightarrow X'$.
\end{itemize}
Then, if $(X', \A') = \bigl(\R, \B(\R)\bigr) \>\, [\,\bigl(\eR, \eB(\eR)\bigr)\,]$,
$${\color{Magenta}u} \text{\it\ is an {\color{Magenta}$(\A/\B [\eB])$-Measurable [Numerical] Function}, } {\color{Magenta}u}\colon X \rightarrow \R \> [\eR]$$
\end{definition}

% 15
\begin{definition}[{[Real-Valued]} Random Variable]\label{def:rvariable}
Given 
\begin{itemize}
\item
A \dref{def:pspace}[\emph{Probability Space}], $(\Omega, \A, p)$, and a \dref{def:mablespace}[\emph{Measurable Space}], $(X[=\R], \A')$,
\item
An \dref{def:mmap}[\emph{$(\A/\A')$-Measurable Mapping}], ${\color{Magenta}Y}\colon\Omega \rightarrow X[= \R]$.
\end{itemize}
Then {\color{Magenta}$Y$} is a {\color{Magenta} \emph{[Real-Valued] Random Variable}}.  The \dref{def:imeasure}[\emph{Image Measure}] of $p$ under $Y$ gives the {\color{Magenta}\emph{Probabilities}} of the {\color{Magenta}\emph{Events,} $E$} $\in \A'.$ 
\end{definition}

% 16
\begin{definition}[Simple Function]\label{def:sfun}
Given
\begin{itemize}
\item
	A \dref{def:mablespace}[\emph{Measurable Space}], $(X, \A)$ along with $\measurespace$,
\item
	An \dref{def:mfun}[\emph{$(\A, \B)$-Measurable Function}], $u: X \rightarrow \R.$
\end{itemize}
Then $u$ is a {\color{Magenta}\emph{Simple Function}} $\logeq$
$$\exists N \in \N: \> \exists(A_j)_{j=0}^N \text{\it\ with } A_j \in \A \text{\it\ and } \> \exists(c_j)_{j=0}^N \text{\it\ with } %
c_j \in \R \> : u(x) = \sum_{j=0}^N c_j \cdot \indi[A_j] \;\; \forall x \in X$$.
Given a measure, $\mu$, on the space $(X, \A)$; {\color{Magenta}\emph{the set of all Simple Functions}} on $(X, \A, \mu)$ is {\color{Magenta}\emph{$\E_\mu$}}. {\color{Magenta}\emph{the set of all non-negative Simple Functions}} is {\color{Magenta}\emph{$\E_\mu^+$}}.
\end{definition}

% 17
\begin{definition}[Integral of a Simple Function]\label{def:sint}
Given
\begin{itemize}
\item
	A \dref{def:mspace}[\emph{Measure Space}], $(X, \A, \mu)$,
\item
	A non-negative \dref{def:sfun}[\emph{Simple Function}], $s\colon X \rightarrow \R_{\geq 0}$,
\item
	$N \in \N$,
\item
	Sets, $(A_j)_{j=0}^N$ and $(c_j)_{j=0}^N : s(x) = \sum_{j=0}^N c_j \cdot \indi[A_j] \;\; \forall x \in X$.
\end{itemize}
Then {\color{Magenta}$I_\mu$(s)} is the {\color{Magenta}\emph{Integral of the Simple Function}}, $s$, where
$${\color{Magenta}I_\mu(s)} = \sum_{j=0}^N c_j \cdot \mu(A_j)$$
\end{definition}